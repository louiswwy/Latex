\section{Des normes de téléphonie mobile}
Depuis 1984, il y a déjà plusieurs standards ont été utilisé par les opérateur dans le monde entier. Voici un tableau de différentes standards mobile en Europe et ses paramétrés\ref{tbl:GMIE}. 
\begin{table}[H]
\begin{tabular}{|p{2cm}|p{2cm}|p{2cm}|p{6cm}|}
	\hline
	Génération&Acronyme&Description&Débit\\
	\hline
	1G		&Radiocom 2000	&Échanges de type voix uniquement&analogique\\
	\hline
	2G		&GSM			&Échanges de type voix uniquement	&9,05 kbps\\
	\hline
	2,5G	&GPRS			&Échange de données sauf voix		&171,2 kbps / 50 kbps / 17,9 kbps\\
	\hline
	2,75G	&EDGE			&Basé sur réseau GPRS existant		&384 kbps / 64 kbps / -\\
	\hline
	3G		&UMTS			&Voix + données						&144 kbps rurale, 384 kbps urbaine, 1,9 Mbps point fixe / -\\
	\hline
	3.5G ou 3G+ ou H&HSPA	&Évolution de l'UMTS				&14,4 Mbps / 3,6 Mbps / -\\
	\hline
	4G		&LTE			&Long Term Evolution				&150 Mbps / 40 Mbps / -\\
	\hline
	4G		&LTE-Advanced	&Long Term Evolution Advanced		&1 Gbps à l'arrêt, 100 Mbps en mouvement / - / -\\
	\hline
\end{tabular}
\caption{Les différentes générations de téléphonie mobile en Europe}
 \label{tbl:GMIE}
\end{table}
En télécommunication, 1G est la premier génération des standards pour la téléphonie mobile, il s'agit de la première apparition du réseau de téléphonie mobile, 1G sont des réseau analogiques, peut transmet seulement le voix.
 

\newpage
\subsection{Une sous section}

On peut mettre des mots en \emph{italique}, 
en \textsc{petites Majuscules} ou 
en \texttt{largeur fixe (machine à écrire)}.

Voici un deuxième paragraphe avec une formule mathématique simple : $e = mc^2$.

Un troisième avec des \og guillemet français \fg{}.

\subsubsection{Écrire en anglais}

\foreignlanguage{english}{Do you speak French? Does anybody here speak french?}


\subsection{Lites}

\begin{itemize}
\item Liste classique ;
\item un élément ;
\item et un autre élément.
\end{itemize}
\vspace{\parskip} % espace entre paragraphes

\begin{enumerate}
\item Une liste numéroté
\item deux
\item trois
\end{enumerate}
\vspace{\parskip}

\begin{description}
\item[Description] C'est bien pour des définitions.
\item[Deux] Ou pour faire un liste spéciale.
\end{description}
\vspace{\parskip}


\subsection{Références}

Voici une référence à l'image de la figure \ref{bloghiko} page \pageref{bloghiko} et une autre vers la partie \ref{p2} page \pageref{p2}.

On peut citer un livre\,\up{\cite{lpp}} et on précise les détails à la fin du rapport dans la partie références.


\subsection{Note de bas de page}

Voici une note\,\footnote{Texte de bas de page} de bas de page.
Une deuxième\,\footnotemark{} déclarée différemment.
La même note\,\footnotemark[\value{footnote}].

\footnotetext{Il a deux références vers cette note}


\subsection{Figure}

\begin{figure}[!ht]
    \center
    \includegraphics[]{./images/bloghiko.jpg}
    \caption{BlogHiko | taille original}
    \label{bloghiko}
\end{figure}

\begin{figure}[H]
    \center
    \includegraphics[width=0.5\textwidth]{./images/bloghiko.jpg}
    \caption{BlogHiko | 50\% de la largeur de la page}
\end{figure}


